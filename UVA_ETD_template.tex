%       File: UVA_ETD_template.tex
%     Created: Wed Oct 20 12:58 PM 2021 EDT
%     Last Change: Wednesday, October 20, 2021
%     Author: Ricky Patterson, UVA
% Based on VTthesis_template.tex by Alan Lattimer VT
% and modifications by Carrie Cross, Robert Browder, and LianTze Lim. 
%
% This template is designed to operate with XeLaTeX.
%
% Formatted to (almost*) meet UVA print requirements (even though this is not  
% necessary for LIBRA ETD upload, it is necessary for a print copy 
% from UVA PCS if you want one, and it does help it look better as an ETD): 
% https://graduate.as.virginia.edu/sites/graduate.as.virginia.edu/files/thesisPhysicalStandards.pdf
%
% *One exception from PCS print requirements - this template assumes you are
% producing a double sided document. Since the print copy is no longer being
% produced for archival purposes, printing double-sided seems acceptable/desirable.
% This means page numbers are at the top outside corner of a page, and the 1.5 inch
% margin is on the inside margin of the page. It works as a normal (double-sided) book.
%
% Title Page meets GSAS ETD requirements: 
% https://graduate.as.virginia.edu/thesis-submission-and-graduation 
% (actual template is here: 
% https://graduate.as.virginia.edu/sites/graduate.as.virginia.edu/files/thesisTitlePageTemplate-2017.pdf)
%
%Further instructions for using this template are embedded in the document. Additionally, there are comments at the end of the file that give suggestions on writing your thesis.
%
%In addition to the standard formatting options, the following options are defined for the UVAthesis class: doublespace, draft. 
% nopageskip - Removes arbitrary blank pages.
\documentclass[doublespace,nopageskip]{UVAthesis} 

% Using the following header instead will create a draft copy of your thesis
%\documentclass[doublespace,draft]{UVAthesis}

% The lipsum package is just included to put dummy "Lorem ipsum dolor sit amet" text in the document in order to demonstrate page headers and table of contents behavior. You should remove it once you begin writing your actual thesis or dissertation.
\usepackage{lipsum}

% Title of your thesis
\title{Your Title $\mu$ Goes Here}

% Your Full Legal Name, including middle
\author{Kendra Marie Smith}

% Your Home Town and State
\homeaddress{Eureka Springs, Kansas}

% Your Previous Degree(s)
\firstpriordegree{Bachelor of Arts}
%\secondpriordegree{Master of Arts}

% Your Previous Institution(s)
\firstpriorinstitution{Eureka Springs State College}
%\secondpriorinstitution{Eureka University}

% Date of Your Previous Degree(s)
\firstpriordate{2012}
%\secondpriordate{2017}

% Change this to your Department, e.g. Physics, Economics, Astronomy, etc.
\program{Astronomy} 

% Change this to your degree, e.g. Master of Science, Master of Art, Doctor of Philosophy
\degree{Master of Science} 

% Change this to the Thesis or Dissertation, depending on your degree type
\thesis{Thesis}

% This should be the date that the degree will be conferred, May, August or December:
\graddate{May 2022} 

% Committee members. One advisor/chair, and up to five additional members.
% Leave unneeded entries commented out.
% 
\advisor{Advisor A. Advisor}
\secondmember{Second X. Member}
\thirdmember{Third X. Member}
\fourthmember{Fourth X. Member}
\fifthmember{Fifth X. Member}
%\sixthmember{Fifth Committee Member}

% The dedication and acknowledgement pages are optional. Comment them out to remove them.
\dedication{This is where you put your dedications.}
\acknowledge{This is where you put your acknowledgments.}

% Using biblatex. This is the standard bibtex file. You need to include the .bib extension in <bib_file_name.bib>.
%\addbibresource{biblatex.bib}
\bibliography{biblatex}

% The abstract is required.
\abstract{Give a brief description of your thesis here.}
\abstract{$\mu = \Sigma^{2-e}$ \lipsum [1-4]}


\begin{document}
% The following lines set up the front matter of your thesis or dissertation and are required to ensure proper formatting per the UVA ETD standards. 
  \frontmatter
  \maketitle
  \tableofcontents

% 
	\listoffigures
	\listoftables
    %\printnomenclature %Creates a list of abbreviations. Comment out to remove it. 

% sample text for abbreviations:
 
\nomenclature{NLP}{Natural Language Processing}
 
 
\nomenclature{$M_\odot$}{The mass of the Sun ($1.989 \times 10^{30}$ kg)}

% The following sets up the document for the main part of the thesis or dissertation. Do not comment out or remove this line.
	\mainmatter

	%now go ahead and start writing your thesis
	\chapter{Introduction} \label{ch:introduction}
   	\lipsum[1-10]		
      
%Copy/paste the code below to add sections and subsections to each chapter. Add your own text to the chapter and (sub)section labels to create custom headings.          
    \section{One Section} \label{se:one_section}
		\lipsum[4-5]	
			\subsection{A sub-section} \label{ss:this_subsection}
			\lipsum[11-14]	
		\section{Another Section} \label{se:another_section}
		\lipsum[17-18]	

    \chapter{Review of Literature} \label{ch:lit_review}
    \lipsum[2-4]
	\chapter{Results} \label{ch:results}
	\lipsum[9-11]
	\chapter{Discussion} \label{ch:discussion}
	\lipsum[6-12]
	\chapter{Conclusions} \label{ch:conclusions}
	\lipsum[5-11]
	\chapter{Summary} \label{ch:summary}
	\lipsum[1-3]
	
	This document is an example of \texttt{biblatex} package use in bibliography 
        management.
        Three items are cited: \textit{The \LaTeX\ Companion} book \cite{latexcompanion}. %\autocite{latexcompanion},
        the Einstein journal paper %\parencite{einstein},
        and Donald Knuth's website %\parencite{knuthwebsite}.
        The \LaTeX\ related items are by %\textcite{latexcompanion,knuthwebsite}. 
        

	% Uncomment the following lines to include your bibliography: 
	\printbibliography[heading=bibintoc]
	   

	% This formats the chapter name to appendix to properly define the headers:
	\appendix

	% Add your appendices here. You must leave the appendices enclosed in the appendices environment in order for the table of contents to be correct.
	\begin{appendices}
		\chapter{The Title of the First Appendix}
		\label{app:appendixA}
		\lipsum[3]
			\section{Section 1 of Appendix A} \label{app:appendixA_1}
			\lipsum[4]
				
			\section{Section two} \label{app:appendixA_2}
			\lipsum[5]
				
		\chapter{Title} 
		\label{app:appendixB}
			\lipsum[6-7]
	\end{appendices}

\end{document}


%****************************************************************************
% Below are some general suggestions for writing your dissertation:
%
% 1. Label everything with a meaningful prefix so that you
%    can refer back to sections, tables, figures, equations, etc.
%    Usage \label{<prefix>:<label_name>} where some suggested
%    prefixes are:
%			ch: Chapter
%     		se: Section
%     		ss: Subsection
%     		sss: Sub-subsection
%			app: Appendix
%     		ase: Appendix section
%     		tab: Tables
%     		fig: Figures
%     		sfig: Sub-figures
%     		eq: Equations
%
% 2. The UVAthesis class provides for natbib citations. You should upload
%	 one or more *.bib bibtex files. Suppose you have two bib files: some_refs.bib and 
%    other_refs.bib.  Then your bibliography line to include them
%    will be:
%      \bibliography{some_refs, other_refs}
%    where multiple files are separated by commas. In the body of 
%    your work, you can cite your references using natbib citations.
%    Examples:
%      Citation                     Output
%      -------------------------------------------------------
%      \cite{doe_title_2016}        [18]
%      \citet{doe_title_2016}       Doe et al. [18]
%      \citet*{doe_title_2016}      Doe, Jones, and Smith [18]
%
%    For a complete list of options, see
%      https://www.ctan.org/pkg/natbib?lang=en
%
% 3. Here is a sample table. Notice that the caption is centered at the top. Also
%    notice that we use booktabs formatting. You should not use vertical lines
%    in your tables.
% 
%				\begin{table}[htb]
%					\centering
%					\caption{Approximate computation times in hh:mm:ss for full order 						versus reduced order models.}
%					\begin{tabular}{ccc}
%						\toprule
%						& \multicolumn{2}{c}{Computation Time}\\
%						\cmidrule(r){2-3}
%						$\overline{U}_{in}$ m/s & Full Model & ROM \\
%						\midrule
%						0.90 & 2:00:00 & 2:08:00\\
%						0.88 & 2:00:00 & 0:00:03\\
%						0.92 & 2:00:00 & 0:00:03\\
%						\midrule
%						Total & 6:00:00 & 2:08:06\\
%						\bottomrule
%					\end{tabular}
%					\label{tab:time_rom}
%				\end{table}
% 
% 4. Below are some sample figures. Notice the caption is centered below the
%    figure.
%    a. Single centered figure:
%					\begin{figure}[htb]
%						\centering
%						\includegraphics[scale=0.5]{my_figure.eps}
%						\caption{Average outlet velocity magnitude given an average  
%				        input velocity magnitude of 0.88 m/s.} 
%						\label{fig:output_rom}
%					\end{figure}
%    b. Two by two grid of figures with subcaptions
%					\begin{figure}[htb]
%						\centering
%						\begin{subfigure}[h]{0.45\textwidth}
%							\centering
%							\includegraphics[scale=0.4]{figure_1_1.eps}
%							\caption{Subcaption number one}
%							\label{sfig:first_subfig}
%						\end{subfigure}
%						\begin{subfigure}[h]{0.45\textwidth}
%							\centering
%							\includegraphics[scale=0.4]{figure_1_2.png}
%							\caption{Subcaption number two}
%							\label{sfig:second_subfig}
%						\end{subfigure}
%
%						\begin{subfigure}[h]{0.45\textwidth}
%							\centering
%							\includegraphics[scale=0.4]{figure_2_1.pdf}
%							\caption{Subcaption number three}
%							\label{sfig:third_subfig}
%						\end{subfigure}
%						\begin{subfigure}[h]{0.45\textwidth}
%							\centering
%							\includegraphics[scale=0.4]{figure_2_2.eps}
%							\caption{Subcaption number four}
%							\label{sfig:fourth_subfig}
%						\end{subfigure}
%						\caption{Here is my main caption describing the relationship between the 4 subimages}
%						\label{fig:main_figure}
%					\end{figure}
%
%----------------------------------------------------------------------------
%
% The following is a list of definitions and packages provided by UVAthesis:
%
% A. The following packages are provided by the UVAthesis class:
%      amsmath, amsthm, amssymb, enumerate, natbib, hyperref, graphicx, 
%      tikz (with shapes and arrows libraries), caption, subcaption,
%      listings, verbatim
%
% B. The following theorem environments are defined by UVAthesis:
%      theorem, proposition, lemma, corollary, conjecture
% 
% C. The following definition environments are defined by UVAthesis:
%      definition, example, remark, algorithm
%
%----------------------------------------------------------------------------






